\documentclass[hidelinks,12pt]{article}
\usepackage{hyperref}
\usepackage[utf8]{inputenc}
\usepackage{titlesec}
\usepackage{polski}
\usepackage[margin=25mm]{geometry}
\linespread{1}
\titleformat*{\section}{\fontsize{14pt}{2}\bfseries}
\titleformat*{\subsection}{\fontsize{13pt}{2}\bfseries}
\titleformat*{\subsubsection}{\fontsize{13pt}{2}\bfseries}
\usepackage[T1]{fontenc}
\usepackage{tgtermes}
\usepackage{amsmath}
\numberwithin{figure}{section} % numeruj rysunki: nr rusunku, nr rozdziału
\numberwithin{equation}{section} % numeruj równania w obebie sekcji
\numberwithin{table}{section} % numeruj tabele w obebie sekcji
\usepackage{dirtree}
\usepackage[]{algorithm2e}
\usepackage{pdfpages}
\SetKwInput{KwData}{Dane}
%\SetAlgorithmName{Algorytm}{algorytm}{Lista algorytmów}
\SetKwInput{KwResult}{Wynik}
\usepackage[parfill]{parskip}
\usepackage{graphicx}
\usepackage{float}
\usepackage{wrapfig}
\usepackage{listings}
\usepackage{subcaption}
\usepackage{multirow} % żeby się nie wykrzaczało na \backslashbox
\usepackage{tabularx} % rozszerzon obsługa tabel
%\usepackage{diagbox} % obsługa przekątnych w tabelach
%\newcolumntype{R}{>{\raggedleft\arraybackslash}X} % definicja tabeli rozciągniętej do szerokości strony
\graphicspath{ {gfx/} }

%\setlength\parindent{24pt} % wcięcia akapitów
%\usepackage{indentfirst} % wcięcia pierwszych akapitów w rozdziale

\usepackage{amsmath}

\newcommand{\fig}[3]{
\begin{figure}
\centering
\includegraphics[ width=\textwidth]{#1}
\caption{#2}
\label{#3}
\end{figure}
}

\author{inż. Jacek Skoczylas 203239 \\inż. Tomasz Wawrzyniak 203341}
\title{Projekt Konduktor\\
Rozwiązanie problemu Konduktora techniką programowania dynamicznego}
\frenchspacing

\begin{document}
\maketitle
\newpage
\tableofcontents
\newpage

\section{Wstęp}
Projekt polegał na wybraniu tematu, opracowaniu modelu, a następnie zaimplementowaniu aplikacji z wykorzystaniem metody programowania dynamicznego.
Dodatkowo rozwiązanie to zostało porównane z analogicznym problemem rozwiązanym metodą brutalfore. Osteteczne wnioski zostały przedstawione w ostatnim punkcie pracy.

\section{Analiza problemu}
Problem konduktora polega na optymalnym doborze odcinków na których odbędzie się kontrola biletów. Z punktu widzenia PKP najlepiej by kontrol była by wykonywana najczęściej tak by skontrolować wszystkie bilety. Niestety dokonywanie kontroli na każdym przystanku nie jest jest możliwe z powodów czasowych. Konduktor musiał by wielokrotnie przechodzić przez cały pociąg oraz ponownie weryfikować bilety, co więcej musiał by też mieć czas na sprzedaż biletów. Dlatego też należy przeprowadzać z góry założoną liczbę kontroli na odpowiednio dobranych odcinkach  tak by maksymalizować liczbę sprawdzonych kontroli. Dodatkowo dane statystyczne ilości wchodzących i wychodzących klientów są znane oraz dostarczane na bierząco przez zewnętrzny system.

\subsection{Model matematyczny}
un - numer kolejnej stacji, takiej że tuż po jej opuszczeniu nastąpi kontrola biletów, ciąg un jest ciągiem rosnącym
xn - numer poprzedniej stacji w jakiej następowała kontrola biletów
N - liczba kontroli biletów do wykonania na całej trasie
yn - ilość skontrolowanych biletów, które nie były wcześniej skontrolowane w ostatniej kontroli

Liczbę pasażerów, którzy jadą od danej stacji do danej stacji opisuje macierz X, w której Xi,j oznacza liczbę pasażerów jaka wsiadła na stacji i i wysiada na stacji j
un, xn, N, yn, Xi, jLiczby naturalne nieujemne, dla każdego i, j.

xn+1=un
yn = i = xnunj = unnXi, j


\subsection{Dane wejściowe-wyjściowe}
Dane wejściowe:
n - liczba kontroli która musi zostać przeprowadzona
X - macierz zawierająca dane z systemu

\section{Prezentacja aplikacji}


\section{Porównanie z techniką brutalforce}

\section{Wnioski}

\end{document}